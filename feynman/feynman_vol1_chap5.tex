\documentclass[10pt,letterpaper]{article}
\usepackage[utf8]{inputenc}
\usepackage{amsmath}
\usepackage{amsfonts}
\usepackage{amssymb}
\author{David Mummy}
\title{Feynman Lectures, Chapter 5: Time and Distance}
\begin{document}
\maketitle
Galileo (5-1) instigated the transition from a philosophical to experimental method of scientific inquiry. In the experiment described here, he used his pulse to measure the position of a ball on a track and discovered that the position of the ball was proportional to the square of the elapsed time: $D \propto t^2$. If we are going to study motion, we must be able to answer questions of where, and of when. \\
The exact definition of time (5-2) is elusive, but we can measure it using something periodic like a day or an hourglass; we can measure shorter times (5-3) using electronic pendulums called \textit{oscillators}. We can measure still shorter times using \textit{distance}: Feynman uses the example of a car turning on its headlights at point A and off at point B. If we know how fast the car was going, we can figure out the elapsed time between the two events (note that this is a somewhat philosophically different way of defining time, but it stands up to our common sense understanding of what time is). \\
\indent Long times (5-4) are measured in obvious ways using days and years, and in less obvious ways such as radioactive decay: if we know a material began by containing a certain amount A of a radioactive isotope, and it now contains an amount B, and we know the half-life $T$ of that isotope -- then the elapsed time $t$ can be found by 
$$ \left(\frac{1}{2}\right)^{t/T} = \frac{B}{A} $$
This is easy with carbon because $\textrm{C}^{14}$ is continuously replenished in the atmosphere, so when an organism dies this proportion gets ``fixed", and we know its half-life is ~5000 years. For longer times, we can also use uraniam (half-life $10^9$ years), as long as we know the original proportion, which often we do.\\
\indent What about large distances (5-6). We can of course measure large distances by counting lots of smaller distances, like a yard stick, but this only works to a certain point. We can also use angles of stellar objects to get relative distances, and then if we can lock down one of those distances (say using a laser), we can lock down the whole network. Beyond that we can use parallax; we can also use our knowledge of the relationship between color and brightness of stars, for if a star of a certain color is dimmed a certain amount, we can determine its distance. We can also guess that galaxies of our same shape are also our same size, and thus infer their distance. \\
\indent What about short distances (5-7)? We can use x-ray crystallography to determine structurefrom scattering patterns. We can also use the idea of an \textit{effective cross section} $\sigma = \pi r^2$, here of a nucleus, and shoot neutrons at a material to see how many bounce back from the nucleus (the neutrons, having no charge, can penetrate the electron cloud). If we know there are $N$ atoms in our material, the proportion ``reflected" is equal to the ratio of $N\sigma$ to the overall surface area of the material, assuming the material is sufficiently thin that the nuclei do not obscure one another along path of the oncoming neutrons. \\
\indent We close by recapitulating ideas of complementarity: uncertainty of position vs. momentum
$$ \Delta x = h/\Delta p $$
and uncertainty of time vs. energy
$$ \Delta t = h/\Delta E $$ where $h$ is of course Planck's constant. 


\end{document}
