\documentclass[10pt,letterpaper]{article}
\usepackage[utf8]{inputenc}
\usepackage{amsmath}
\usepackage{amsfonts}
\usepackage{amssymb}
\author{David Mummy}
\title{Feynman Lectures, Chapter 4: Conservation of Energy}
\begin{document}
\maketitle
\indent Feynman talks about the general concept of the law of \textit{conservation of energy} (4-1), using the allegory of a child with a bunch of blocks. The number of blocks is always the same, but sometimes we have to do some fancy counting to understand \textit{how} it is always the same: one is in a box, or under a rug, or in the tub full of dirty water and all we can see is that the water level has risen, or a friend came over and left an extra one. But the point is that if we have a sufficient way of counting, we can always account for all the energy in a closed system (or an open system if we are able to account for energy going in and out), even though it may change form. These forms are, in particular, gravitational, kinetic, heat, elastic, electrical, chemical, radiant, nuclear, and mass energy. What exactly energy is, we don't really know, but if one of these quantities decreases in our system, we know it has reappeared elsewhere. \\ 
\indent \textit{Gravitational potential energy} (4-2) is the energy of position in a gravitational field. We posit that perpetual motion is impossible, and thereby through the use of ``perfectly reversible machines" involving levels and weights, and some logic, determine that a reversible machine is in fact \textit{the best machine}, and any other machine is necessarily inferior (otherwise we could combine it with a reversible machine and get endless energy. We can extend this argument a bit further using these simple machines and arrive at the formula for gravitational potential energy (close to the earth), namely
$$
(\textrm{grav. pot. energy}) = (\textrm{weight})  \times (\textrm{height}).
$$
This is gravitational potential energy, but we can generalize this to plain old potential energy as 
$$
\Delta E = (\textrm{force})  \times  (\textrm{distance force acts through}).
$$
Using this idea we can determine properties of all kinds of machines, as long as they're reversible (so, in reality, none, but close): the force needed on a jackscrew to lift a car, or the force needed on the end of a lever with weights positioned at certain locations. All we need to do is look at the balance of competing force-distances.

\indent A different form of energy, not one of position but of motion, is (you guessed it!) \textit{kinetic energy} (4-3). We now consider a pendulum at rest (but before we drop it). It's not moving, so it's energy is just one of position, and all this energy is ``used up" once we drop it and it hits the bottom of its arc. However, it is able to recover all that positional energy at the other end of its arc as it rises again (it being, of course, perfectly reversible). Where has this energy gone? It is in the kinetic energy. So the original gravitational potential of weight $w$ times height $h$ (the height it was at before we dropped it) must equal the kinetic energy at the bottom of the arc: $w \times h = KE$. We don't derive the kinetic energy here but get it as \textit{fait accompli}:
$$
\textrm(K.E.) = \frac{wv^2}{2g}
$$
where $v$ is of course velocity and $g$ the acceleration due to gravity. This is a bit different than the traditional $\frac{1}{2}mv^2$ but recall that $w$ is just $mg$ and it works out. 

\indent There are other forms of energy (4-4), of course, as alluded to earlier, including heat energy (really just jumbled up kinetic energy), chemical, nuclear, mass, radiant, and so on, as foreshadowed earlier. There are also other \textit{laws of conservation}: \textit{conservation of linear momentum} and \textit{conservation of angular momentum}. Feynman here makes the tantalizing statements that time independence (i.e. nothing depends on an absolute measure of time) plus quantum mechanics gives us conservation of energy; that spatial invariance plus quantum machanics yields conservation of linear momentum; and that rotational invariance plus quantum mechanics yields conservation of angular momentum. How exactly this all fits together is unclear, and presumably to be filled it to the more sophisticated reader of chapters still forthcoming.
\indent There are also some conservation laws that involve actual counting, rather than the continuous measures of energy and momentum: \textit{charge} is conserved, as are \textit{baryons} and \textit{leptons}. Together, these make the six conservation laws. 

\indent Just because energy is conserved does not, of course, mean that it is conserved in any sort of usable form. The laws which govern available energy and entropy are the province of \textit{thermodynamics}. 


\end{document}
