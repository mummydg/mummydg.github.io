\documentclass[10pt,letterpaper]{article}
\usepackage[utf8]{inputenc}
\usepackage{amsmath}
\usepackage{amsfonts}
\usepackage{amssymb}
\author{David Mummy}
\title{Feynman Lectures, Chapter 2: Basic Physics}
\begin{document}
\maketitle
We begin (1-1) with a summary of the essence of scientific thought -- that extraordinarily complex phenomena can be boiled down and in some sense explained by a set of simpler rules: the idea of reductivity. The scientific method facilitating this is one of \textit{observation, reason, and experiment}, to try to understand the ``rules of the game." Feynman characterizes three different ways of generating and assessing guesses about the rules:
\begin{itemize}
\item \textbf{Simplify}. Can we find (or create) simple situations where we can cut through the noise and get closer to the rules in isolation?
\item \textbf{Generalize}. Find overall patterns. Perhaps we do not know exact individual rules, but we can at least generate a rule that fits observation, even if it is not entirely satisfying: bishops stay on the same color squares, even if we do not know how they move. 
\item \textbf{Approximate}. ``The Most Powerful Of Them All." Overall, what is happening, and what are the net effects? This rule may not fit the observed data exactly, but it gives us a sense of what is going on. The difference between this and a generalization is somewhat subtle. 
\end{itemize}
Physics before 1920 (2-2) lays out the concept of the world in the pre-quantum age: a three-dimensions space comprised of particles changing with time, and the early concept of the forces -- short-range forces (electromagnetic and gravity, at the time). Electrical charges are almost completely balanced out; the EM force is tremendously strong compared to gravity. Atoms have a positively charged, massive nucleus, surrounded by the electrons that determine its chemical properties. \\
\indent The electrical force is more complicated than first thought; there is the \textit{electric field}, which is the potential for producing a force, and goes inversely with distance rather than as its inverse square. Thus jiggles in electrons can produce jiggles in other electrons very far away; it's the difference between pushing on water, and jiggling a cork up and down in the water. The water itself is not moving longitudinally in the latter example, just up and down. 
\indent The EM field can carry waves ranging from what Feynman terms an electrical disturbance (on the order of  $10^2$ hertz) to the $\gamma $-rays in cosmic rays ($10^27$ hertz), with radio, visible light, and X-rays in between, and with behavior going from a field to waves to particles. \\
Then along came quantum physics (2-3) to shake everything up. It transpired that momentum and position cannot both be known to an arbitrary degree, and in fact, $\Delta x \Delta p \geq \frac{h}{2\pi}$. This effect (among other things) keeps electrons from collapsing into the nucleus. \\
\indent Along with this came the idea that it is ``\textit{fundamentally impossible} to make a precise prediction of \textit{exactly what will happen} in a given experiment," which is clearly a tremendous shift in our understanding of the workings of the universe. We then return to the idea that the test of any idea is experiment, but that there are many factors that we do not understand may get in the way of reproducibility (Stockholm is not the same as Honolulu) and we simply must take what we get and go from there. \\
\indent Quantum mechanics unifies the wave/particle issue, and we have the \textit{photon} as the medium of interaction between electrons and protons -- \textit{quantum electrodynamics}. Out of QED comes all known electrical, mechanical, and chemical laws (to the extend that we can actually apply them in a complex world). 
\indent Lastly, there is the matter of the nuclei and particles (2-4). We take a quick and somewhat incomprehensible tour through baryons, mesons, leptons, and their antiparticles, and the idea that the existence of particles can be inferred from theory, and we then go look for them. This is all a bit vague and rushed and we come back to it later, so I won't recap it here. 


\end{document}