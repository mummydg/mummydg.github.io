\documentclass[10pt,letterpaper]{article}
\usepackage[utf8]{inputenc}
\usepackage{amsmath}
\usepackage{amsfonts}
\usepackage{amssymb}
\author{David Mummy}
\title{Feynman Lectures, Chapter 1: Atoms in Motion}
\begin{document}
\maketitle
This chapter sets the tone for the upcoming material and is light on equations. Feynman sets forth a few basic principles (1-1): 
\begin{itemize}
\item Our ideas of physical law are by necessity approximations based on observations of nature, and \textit{"the test of all knowledge is experiment."}
\item The single most useful idea in scientific knowledge is that \textit{"all things are made of atoms -- little particles that move around in perpetual motion, attracting each other when they are a little distance apart, but repelling upon being squeezed into one another."}
\end{itemize}
We then (1-2) delve into ideas of temperature and note that water as a solid has a predictable crystalline structure, with the position of one molecule related to the position of distance molecules, in contrast to a liquid or gas. Water is unusual in that its crystalline structure is quite open, meaning it grows denser when it melts. We note in passing that the radius of an atom is on the order of 1-2 \textit{angstroms} ($10^{-10}$m), and the relationship between pressure $P$, force $F$, and area $A$, namely $P=\frac{F}{A}$.

This essential  idea of the kinetic atom (or molecule) (1-3) can be used to understand basic ideas of everyday phenomena: water turning into steam when heated, why blowing a fan across a pool of water causes it to evaporate faster, how temperature and pressure are related, and so on. The most energetic particles in your bowl of soup are most likely to be ejected into the air, so by blowing across the bowl you are effectively selectively eliminating the hot particles! An additional process is a salt dissolving in a liquid; if there is a small amount, the ions separate (there are not proper salt ``molecules," just a big glop of ions in a lattice) due to interactions with the water molecules; if there is too much salt and the water molecules are beyond capacity to pair them off, the salt precipitates into a solid. 

Finally (1-3) we mention chemical reactions and a basic burn of carbon being joined to one or two oxygen atoms, depending on the amount of gas and speed of the reaction (carbon monoxide being the result of "incomplete" combustion). We briefly go over an example of an organic chemist's shorthand description of a more complicated motel, name-drop \textit{Brownian motion}, and recapitulate the main theme that \textit{everything is made of atoms}. Yes, even you! 

\end{document}