\documentclass[10pt,letterpaper]{article}
\usepackage[utf8]{inputenc}
\usepackage{amsmath}
\usepackage{amsfonts}
\usepackage{amssymb}
\author{David Mummy}
\title{Feynman Lectures, Chapter 3: The Relation of Physics to Other Sciences}
\begin{document}
\maketitle
\indent This chapter discusses the role played by physics in other sciences (3-1), although if something is not a science it is not necessarily bad. (Math is not a science, as the test of its validity is not experiment). \\
\indent Chemistry (3-2) is the most deeply affected by physics, including \textit{inorganic chemistry} (physical chemistry: rates at reactions occur, and what is \textit{happening} in detail; and quantum chemistry: what happens in terms of physical laws). \textit{Statistical mechanics} is the study of heat, or thermodynamics: situations where there are a boatload of mechanical effects at play, as is the case in most situations -- far too many for a computer to handle. There is also \textit{organic chemistry}, the study of molecules in biological systems. However, most of organic chemistry sits ``on top of" the previously described disciplines, and is more interested in the analysis and synthesis of biological substances, rather than the more fundamental studies of inorganic chemistry. \\
\indent Biology (3-3) is then given a quick survey, and Feynman takes brief forays into impulse conduction on nerves, how isotopes can be used to track particular molecules, and ideas of enzymes, proteins, and DNA. However, none of this is particularly germane to these notes, and some of the DNA/RNA information is outdated, so the previous summary suffices for these notes. \\
\indent Astronomy (3-4) is introduced as the progenitor for physics, spurred by man's obseverations of stellar and planetary motion. He includes a footnote which is memorable enough to bear repeating in its entirety here:
\begin{quote}
How I'm rushing through this! How much each sentence in this brief story contains. ``The stars are made of the same atoms as the earth." I usually pick one small topic like this to give a lecture on. Poets say science takes away from the beauty of the stars -- mere globs of gas atoms. Nothing is ``mere." I too can see the stars on a desert night, and feel them. But do I see less or more? The vastness of the heavens stretches my imagination -- stuck on this carousel my little eye can catch one-million-year-old light. A vast pattern -- of which I am a part -- perhaps my stuff was belched from some forgotten star, as one is belching there. Or see them with the greater eye of Palomar, rushing all apart from somem common starting point when they were perhaps all together. What is the pattern, or the meaning, or the \textit{why}? It does not do harm to the mystery to know a little about it. For far more marvelous is the truth than any artists of the past imagined! Why do the poets of the present not speak of it? What men are poets who can speak of Jupiter as if he were like a man, but if he is an immense spinning sphere of methane and ammonia must be silent? 
\end{quote}
\indent We can understand the innards of stars from their spectra and our knowledge of statistical mechanics, and we know that material in stars undergo \textit{nuclear reactions}; we can see the results of this in the isotopes of the cold dead ember which is Earth. \\
\indent Weather and geology (3-5) are complex and we don't really get them, at least not in 1963. Next! \\
\indent Psychology (3-6) is also resistent to efforts to understand at a physical level. Feynman bemoans that we can't understand dogs, much less humans; and hints at the possibility of machine learning to mimic a neural network. \\
\indent Finally (3-7), it appears that physics does not have a ``historical question" like that of astronomy, biology, or geology, although we may have one at some point in the future. And, if you can discover a satisfactory theory of circulating or turbulent fluids, you will be rich! This applies to everything from stars to  weather to water in a pipe. We see what happens, but cannot predict it from first principles -- that is, when it is turbulent flow, not when it is smooth, like in your textbook. We close with another paragraph that is worth repeating whole: 
\begin{quote}
A poet once said, ``The whole universe is in a glass of wine." We will probably never know in what sense he meant that, for poets do not write to be understood. But it is true that if we look at a glass of wine closely enough, we see the entire universe. There are the things of physics: the twisting liquid which evaporates depending on the wind and weather, the reflections in the glass, and our imagination adds the atoms. The glass is a distillation of the earth's rocks, and in its composition we see the screts of the universe's age, and the evoluition of stars. What strange array of chemicals are in the wine? How did they come to be? There are the ferments, the enzymes, the substrates, and the products. There in wine is found the great generalization: all life is fermentation. Nobody can discover the chemistry of wine without discovering, as did Louis Pasteur, the cause of much disease. How vivid is the claret, pressing its existence into the consciousness that watches it! If our small minds, for some convenience, divide this glass of wine, this universe, into parts -- physics, biology, geology, astronomy, psychology, and so on -- remember that nature does not know it! So let us put it all back together not forgetting ultimately what it is for. Let it give us one more final pleasure: drink it and forget it all! 
\end{quote}

\end{document}
